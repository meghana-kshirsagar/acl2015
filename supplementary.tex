\PassOptionsToPackage{usenames}{color}
\documentclass[11pt,a4paper]{article}

\usepackage{relsize} % relative font sizes (e.g. \smaller). must precede ACL style
\usepackage{style/acl2015}
\usepackage[colorlinks=true,linkcolor=black,citecolor=black,filecolor=black,urlcolor=black]{hyperref}
\usepackage{natbib}

\usepackage{multirow}
\usepackage{booktabs} % pretty tables
\usepackage{multicol}
\usepackage{footnote}



%\usepackage{times}
%\usepackage{latexsym}

\usepackage{microtype}
\usepackage{ dsfont }


\usepackage{amssymb}	%amsfonts,eucal,amsbsy,amsthm,amsopn
\usepackage{amsmath}

\usepackage{mathptmx}	% txfonts
\usepackage[scaled=.8]{beramono}
\usepackage[scaled=.85]{helvet}
\usepackage[T1]{fontenc}
\usepackage[utf8x]{inputenc}

\usepackage{MnSymbol}	% must be after mathptmx

\usepackage{latexsym}


\newcommand{\ensuretext}[1]{#1}
\newcommand{\cjdmarker}{\ensuretext{\textcolor{mdgreen}{\ensuremath{^{\textsc{CJ}}_{\textsc{D}}}}}}
\newcommand{\nssmarker}{\ensuretext{\textcolor{magenta}{\ensuremath{^{\textsc{NS}}_{\textsc{S}}}}}}
\newcommand{\mkmarker}{\ensuretext{\textcolor{red}{\ensuremath{^{\textsc{M}}_{\textsc{K}}}}}}
\newcommand{\stmarker}{\ensuretext{\textcolor{orange}{\ensuremath{^{\textsc{S}}_{\textsc{T}}}}}}
\newcommand{\nasmarker}{\ensuretext{\textcolor{blue}{\ensuremath{^{\textsc{NA}}_{\textsc{S}}}}}}
\newcommand{\jbmarker}{\ensuretext{\textcolor{orange}{\ensuremath{^{\textsc{J}}_{\textsc{B}}}}}}
\newcommand{\dbmarker}{\ensuretext{\textcolor{purple}{\ensuremath{^{\textsc{D}}_{\textsc{B}}}}}}
\newcommand{\arkcomment}[3]{\ensuretext{\textcolor{#3}{[#1 #2]}}}
%\newcommand{\arkcomment}[3]{}
\newcommand{\cjd}[1]{\arkcomment{\cjdmarker}{#1}{mdgreen}}
\newcommand{\nss}[1]{\arkcomment{\nssmarker}{#1}{magenta}}
\newcommand{\mk}[1]{\arkcomment{\mkmarker}{#1}{red}}
\newcommand{\st}[1]{\arkcomment{\stmarker}{#1}{orange}}
\newcommand{\jb}[1]{\arkcomment{\jbmarker}{#1}{blue}}
\newcommand{\db}[1]{\arkcomment{\dbmarker}{#1}{purple}}
\newcommand{\nascomment}[1]{\arkcomment{\nasmarker}{#1}{blue}}


\newcommand{\fnf}[1]{\textsc{\textsf{#1}}} % FrameNet frame
\newcommand{\fnr}[1]{\textbf{\textsf{#1}}} % FrameNet role (frame element name)
\newcommand{\fnlu}[1]{\textsf{#1}} % FrameNet lexical unit (predicate)
\newcommand{\pbf}[1]{\mbox{\textsf{#1}}} % PropBank frame (roleset)
\newcommand{\pbr}[1]{\textbf{\textsf{#1}}} % PropBank role (numbered or modifier argument label)
\newcommand{\vpred}[1]{\textbf{#1}} % verb predicate


%\usepackage[top=0.7in, bottom=0.9in, left=0.8in, right=0.8in]{geometry}

\begin{document}

\title{Additional Results}

\maketitle

\section{SemLink}

Bridging across different kinds of lexical resources 
with different levels of coverage can expand the playing field of semantic parsers.
E.g., \citet{shi-05} used a rule-based mapping between FrameNet and VerbNet, and between VerbNet and WordNet, 
%\nss{Also look at: \citep{merlo-09} (multiple resources + levels of generalization), \citep{furstenau-09} (semi-supervised)}
from which they built a rule-based semantic parser.
Here, we aim to expand the training data (token-level) by rule-based mapping 
from distant annotations into the FrameNet representation.



SemLink \citep{bonial-13} is a partial and semi-automatic augmentation
of the PropBank corpus's roleset annotations with mappings to FrameNet and VerbNet.
It contains two types of mappings. The \emph{sense-level mappings} give correspondences between the concepts
from each resource---i.e., between `frames' from FN, `rolesets' from PB, and `roles' from VN.
Since they map different interpretations and granularities
of concepts, the sense-level mappings may be one-to-one, one-to-many, or many-to-many.
PropBank and FrameNet are mapped indirectly via VerbNet. %, introducing additional coverage gaps and ambiguities.
Second, SemLink provides some \emph{token-level parallel annotations}
for the 3~representations in a subset of the PB-WSJ text: hereafter SL-WSJ.

We focus on using token-level SemLink version~1.2.2c annotations as a (disambiguated) mapping from PB to FN tokens.
Of the available 74,977 SL-WSJ verbs, a majority cannot be mapped to FN frames for various reasons.
Around 31\% of the predicates have the frame label \texttt{IN} (``indefinite'') where the mapping from VerbNet to FrameNet is ambiguous.
About 20\% of the instances are labeled \texttt{NF} (``no frame''), indicating a coverage gap in FrameNet.
21\% of verbs have frame labels but no frame element annotations. Most of these are predicates with modifier arguments.
Other arguments pointed to null anaphora that could not be resolved to overt arguments.
This leaves 15,323 mappable instances with at least one overt argument, or 20\% of SL-WSJ verbs. This is a very small
subset of the entire PB annotated data.

Some of the FN information in SemLink is out of date due to subsequent changes in FrameNet.
There are some erroneous FN annotations as well: e.g., all 14~instances of \vpred{liquidate} are
labeled \fnf{Killing}, despite being used in the financial sense; and in 17~cases \vpred{direct}
is erroneously marked as \fnf{Behind\_the\_scenes} (i.e., film direction).
These kind of errors are hard to detect and remove.

\paragraph{$F_1$ on FT test.} Including the SemLink token-level annotations in the training data (FT+SemLink)
causes $F_1$ to drop to 47.9 (by a whopping 11.2\%) below the baseline. 
The guide features, however, modulate the influence of the SemLink annotations, 
giving a minor increase of 0.05\% over the baseline.


\section{Evaluation on exemplars}
We also evaluated the models on a held out exemplar test set, and saw a drastic improvement in $F_1$ for models that use the exemplars. 
However, the distribution of exemplars is skewed in unpredictable ways, so we take this only as a suggestive positive indication.
Table \ref{tbl:results} shows these results. Our best model improves the baseline's $F_1$ by 24.7\%.


\setlength\tabcolsep{4pt}

\begin{table}\centering\small
\def\arraystretch{1}
\begin{tabular}{lrrrr}%r@{~~}rrr}
\toprule
\multicolumn{1}{c}{\textbf{Training Configuration}} 
& \multicolumn{1}{c}{\textbf{Model}} &  \multicolumn{1}{c}{P} &
\multicolumn{1}{c}{R} 
&  \multicolumn{1}{c}{$F_1$} \\
% & \multicolumn{3}{c}{} \\ %&& \multicolumn{3}{c}{\textbf{Exemplars}} \\
%\cline{3-5}%\cline{8-10}
 \multicolumn{1}{c}{\textbf{(Features)}} 
& \multicolumn{1}{c}{\textbf{Size}} 
&  \multicolumn{1}{c}{(\%)} &  \multicolumn{1}{c}{(\%)} &  \multicolumn{1}{c}{(\%)}\\
% && P\hphantom{11} 
% & R\hphantom{11} 
% & $F_1$\hphantom{0} \\
\midrule
%(Baseline) & FT (Basic) & 66.03 & 53.79 & 59.29 && 64.90 & 33.60 & 44.27 \\
FT (Baseline) & 1.1 & 62.63 & 37.65 & 47.03 \\
\midrule
FT (siblings) & 1.9 & 64.81 & 39.09 & 48.77 \\
FT (siblings+parents) & 2.6 & 65.25 & 38.18 & 48.18 \\
\midrule
Exemplars $\xrightarrow{\text{guide}}$ FT & 1.2 & 67.71 & 48.08 & 56.23\\
FT+Exemplars (Basic) & 5.0 & 75.44 & 65.11 & \bf{69.89} \\
FT+Exemplars (DA) & 5.8 & 73.88 & 61.40 & 67.06 \\
\midrule
PB-SRL $\xrightarrow{\text{guide}}$ FT & 1.2 & 61.38 & 39.14 & 47.80 \\
\midrule
\multicolumn{5}{c}{\em{Combining the best methods}}\\
\midrule
PB-SRL $\xrightarrow{\text{guide}}$ FT+Exemplars & 5.5 & 76.14 & 67.71 & \textbf{71.70} \\
FT+Exemplars (siblings) & 9.3 & 77.15 & 65.47 & 70.83 \\
\bottomrule
\end{tabular}
\caption{Argument identification results on the Exemplars test set.
Model size is in millions of features.
\label{tbl:results}}
\end{table}




\bibliographystyle{style/aclnat}
% you bib file should really go here
\setlength{\bibsep}{1pt}
{\fontsize{10}{12.25}\selectfont
\bibliography{argid}}


\end{document}
